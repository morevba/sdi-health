%%%%%%%%%%%%%%%%%%%%%%%%%%%%%%%%%%%%%%%%%
% Beamer Presentation
% LaTeX Template
% Version 1.0 (10/11/12)
%
% This template has been downloaded from:
% http://www.LaTeXTemplates.com
%
% License:
% CC BY-NC-SA 3.0 (http://creativecommons.org/licenses/by-nc-sa/3.0/)
%
%%%%%%%%%%%%%%%%%%%%%%%%%%%%%%%%%%%%%%%%%

%----------------------------------------------------------------------------------------
%	PACKAGES AND THEMES
%----------------------------------------------------------------------------------------

\documentclass{beamer}

\mode<presentation> {
	
	% The Beamer class comes with a number of default slide themes
	% which change the colors and layouts of slides. Below this is a list
	% of all the themes, uncomment each in turn to see what they look like.
	
	%\usetheme{default}
	%\usetheme{AnnArbor}
	%\usetheme{Antibes}
	%\usetheme{Bergen}
	%\usetheme{Berkeley}
	%\usetheme{Berlin}
	%\usetheme{Boadilla}
	%\usetheme{CambridgeUS}
	%\usetheme{Copenhagen}
	%\usetheme{Darmstadt}
	%\usetheme{Dresden}
	%\usetheme{Frankfurt}
	%\usetheme{Goettingen}
	%\usetheme{Hannover}
	%\usetheme{Ilmenau}
	%\usetheme{JuanLesPins}
	%\usetheme{Luebeck}
	\usetheme{Madrid}
	%\usetheme{Malmoe}
	%\usetheme{Marburg}
	%\usetheme{Montpellier}
	%\usetheme{PaloAlto}
	%\usetheme{Pittsburgh}
	%\usetheme{Rochester}
	%\usetheme{Singapore}
	%\usetheme{Szeged}
	%\usetheme{Warsaw}
	
	% As well as themes, the Beamer class has a number of color themes
	% for any slide theme. Uncomment each of these in turn to see how it
	% changes the colors of your current slide theme.
	
	%\usecolortheme{albatross}
	%\usecolortheme{beaver}
	%\usecolortheme{beetle}
	%\usecolortheme{crane}
	%\usecolortheme{dolphin}
	%\usecolortheme{dove}
	%\usecolortheme{fly}
	%\usecolortheme{lily}
	%\usecolortheme{orchid}
	%\usecolortheme{rose}
	%\usecolortheme{seagull}
	%\usecolortheme{seahorse}
	%\usecolortheme{whale}
	%\usecolortheme{wolverine}
	
	%\setbeamertemplate{footline} % To remove the footer line in all slides uncomment this line
	%\setbeamertemplate{footline}[page number] % To replace the footer line in all slides with a simple slide count uncomment this line
	
	%\setbeamertemplate{navigation symbols}{} % To remove the navigation symbols from the bottom of all slides uncomment this line
}

\usepackage{graphicx} % Allows including images
\usepackage{booktabs} % Allows the use of \toprule, \midrule and \bottomrule in tables

%----------------------------------------------------------------------------------------
%	TITLE PAGE
%----------------------------------------------------------------------------------------

\title[Caseload Analysis]{SDI - Caseload Results} % The short title appears at the bottom of every slide, the full title is only on the title page

\author{Michael Orevba} % Your name
\institute[The World Bank] % Your institution as it will appear on the bottom of every slide, may be shorthand to save space
{
	The World Bank Group \\ % Your institution for the title page
	\medskip
}
\date{\today} % Date, can be changed to a custom date

\begin{document}
	
	\begin{frame}
		\titlepage % Print the title page as the first slide
	\end{frame}
	
	\begin{frame}
		\frametitle{Overview} % Table of contents slide, comment this block out to remove it
		\tableofcontents % Throughout your presentation, if you choose to use \section{} and \subsection{} commands, these will automatically be printed on this slide as an overview of your presentation
	\end{frame}
	
	%----------------------------------------------------------------------------------------
	%	PRESENTATION SLIDES
	%----------------------------------------------------------------------------------------
	
	%------------------------------------------------
	\section{First Section} % Sections can be created in order to organize your presentation into discrete blocks, all sections and subsections are automatically printed in the table of contents as an overview of the talk
	%------------------------------------------------
	
	\subsection{Overall Caseload Stats} % A subsection can be created just before a set of slides with a common theme to further break down your presentation into chunks


	%----------------------------------------------------------------------------------------
	%	Introduction  Slides 
	%----------------------------------------------------------------------------------------
	
	\begin{frame}
		\frametitle{Measurement of Caseload}
		Caseload is defined as the number of outpatient visits per healthcare provider per day of each facility. It is calculated based on the number of outpatient visits within the last three months, the number of days per week that the facility is open, and the number of healthcare providers (including all types of providers that regularly offer outpatient consultations). The denominator is adjusted for the presence of providers to reflect the total number of healthcare providers available to provide care.
		
	\end{frame}
	
	%------------------------------------------------


		\begin{frame}
		\frametitle{Overall Caseload}
			\begin{figure}[H] 
				\centering
					\caption{Average Caseload by Country} 
				\includegraphics[width=.7\textwidth]{"../Output/Final/bar_gr_caseload_country"}
			\end{figure}
		\end{frame}

%------------------------------------------------
	
	\begin{frame}
		\frametitle{Overall Caseload}
			\begin{figure}[H] 
				\centering
				\caption{Average Caseload by Country Grouping} 
				\includegraphics[width=.7\textwidth]{"../Output/Final/bar_gr_caseload_group"}
			\end{figure}
	\end{frame}
	
	%------------------------------------------------
	
	%------------------------------------------------
	\section{Second Section}
	\subsection{Breakdown of Caseload by Density} 
	%------------------------------------------------
	
	%------------------------------------------------
	
		\begin{frame}
			\frametitle{Caseload Density}
			\begin{figure}[H] 
				\centering
				\caption{Average Caseload by Facility Level} 
				\includegraphics[width=.7\textwidth]{"../Output/Final/k_den_caseload_fl"}
			\end{figure}
		\end{frame}
	
	%------------------------------------------------
	
	\begin{frame}
		\frametitle{Caseload Density}
		\begin{figure}[H] 
			\centering
			\caption{Average Caseload by Geography} 
			\includegraphics[width=.7\textwidth]{"../Output/Final/k_den_caseload_region"}
		\end{figure}
	\end{frame}

	%------------------------------------------------

	\begin{frame}
		\frametitle{Caseload Density}
		\begin{figure}[H] 
			\centering
			\caption{Average Caseload by Geography without Kenya \& Nigeria} 
			\includegraphics[width=.7\textwidth]{"../Output/Final/k_den_caseload_region_wKN"}
		\end{figure}
	\end{frame}
	
	%------------------------------------------------
	\section{Third Section}
	\subsection{Breakdown of Caseload by Percentiles} 
	%------------------------------------------------
	
		
	\begin{frame}
		\frametitle{Caseload Percentiles}
		\begin{figure}[H] 
			\centering
			\caption{Box Plot Percentiles by Country} 
			\includegraphics[width=.7\textwidth]{"../Output/Final/box_plot_caseload_country"}
		\end{figure}
	\end{frame}
	
	
	%------------------------------------------------
	
	\begin{frame}
		\frametitle{Caseload Percentiles}
		\begin{figure}[H] 
			\centering
			\caption{Box Plot Percentiles by Facility Level} 
			\includegraphics[width=.7\textwidth]{"../Output/Final/box_plot_caseload_fl"}
		\end{figure}
	\end{frame}
	
	
	%------------------------------------------------
	
	\begin{frame}
			\frametitle{Caseload Percentiles}
			\begin{figure}[H] 
				\centering
				\caption{Box Plot Percentiles by Facility Geography} 
				\includegraphics[width=.7\textwidth]{"../Output/Final/box_plot_caseload_fl_region"}
			\end{figure}
	\end{frame}
	
		%------------------------------------------------
	
	\begin{frame}
		\frametitle{Caseload Percentiles}
		\begin{figure}[H] 
			\centering
			\caption{Box Plot Percentiles by Geography} 
			\includegraphics[width=.7\textwidth]{"../Output/Final/box_plot_caseload_region"}
		\end{figure}
	\end{frame}

		%------------------------------------------------
	
		\begin{frame}
		\frametitle{Caseload Percentiles}
			\begin{figure}[H] 
				\centering
				\caption{Box Plot Percentiles by Geography without Kenya \& Nigeria } 
				\includegraphics[width=.7\textwidth]{"../Output/Final/box_plot_caseload_region_wKN"}
			\end{figure}
	\end{frame}

	
	%------------------------------------------------
	
	\begin{frame}
		\frametitle{Next Steps}
			\textbf{Research Focus:}
					\begin{enumerate}
						\item What other facts about caseload do we want to show?
						\item Are there other factors related to caseload that we should investigate?
					\end{enumerate}
	\end{frame}
	
	%------------------------------------------------
	
	\begin{frame}
		\Huge{\centerline{The End}}
	\end{frame}
	
	%----------------------------------------------------------------------------------------
	
\end{document} 